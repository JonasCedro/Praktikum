\documentclass{beamer}

\mode<presentation> {

\usetheme{Madrid}

}
\usepackage{graphicx} % Allows including images
\usepackage{booktabs} % Allows the use of \toprule, \midrule and \bottomrule in tables
\usepackage{amsmath}
\usepackage[utf8]{inputenc} 
\usepackage[T1]{fontenc} 
\usepackage{lmodern} 
\usepackage{ngerman}
\usepackage{amsmath}
\usepackage{eurosym}
\usepackage{glossaries-extra}


%----------------------------------------------------------------------------------------
%	TITLE PAGE
%----------------------------------------------------------------------------------------

\title[Copulas]{Multivariate probability Distributions: Copulas} % The short title appears at the bottom of every slide, the full title is only on the title page

\author{Alexander Ritz} % Your name
\institute[University of Göttingen] % Your institution as it will appear on the bottom of every slide, may be shorthand to save space
{
University of Göttingen \\ % Your institution for the title page
\medskip

}
\date{\today} % Date, can be changed to a custom date

\begin{document}

\begin{frame}
\titlepage % Print the title page as the first slide
\end{frame}

\begin{frame}
\frametitle{Overview} % Table of contents slide, comment this block out to remove it
\tableofcontents % Throughout your presentation, if you choose to use \section{} and \subsection{} commands, these will automatically be printed on this slide as an overview of your presentation
\end{frame}



%------------------------------------------------
\section{Idea} 
%------------------------------------------------
\frame{\sectionpage}

\begin{frame}
\frametitle{Preliminaries}

In general, a multivariate probability distribution is described by its cumulative distribution function (cdf). For simplicity's sake, we will mostly restrict ourselves to real valued and continuous RVs in this introdution.\\
Let $\textbf{X} = (X_1, ..., X_n) $ be an n-dimensional random variable such that:
\[\textbf{X}: \Omega \rightarrow \mathbb{R}^n : \textbf{X}(\omega)=(X_1(\omega),..., X_n(\omega))'\]
with probability space
\[(\Omega, \mathcal{F}, P)\] or specifically \[(\mathbb{R}^n, \mathcal{B}^n, P_X)\]

\end{frame}


\begin{frame}
\frametitle{Preliminaries}

Then \textbf{X} can be described by the joint cdf of $X_1, ..., X_n$:
\[F_{\textbf{X}}(x_1, ..., x_n)= P_{\textbf{X}}(X_1 \leq x_1, ..., X_n \leq x_n)\]
with marginal cdf's
\[F_{X_1}(x_1) = P_{X_1}(X_1 \leq x_1), ..., F_{X_n}(x_n)=P_{X_n}(X_n \leq x_n)\]
\end{frame}


\begin{frame}
\frametitle{Preliminaries}

Recall the properties of the following transformations:
\[\text{ }1. \text{ If  }U \sim \mathcal{U} (0, 1) \text{ and } F \text{ is a cdf, then:} \quad X = F^{-1}(U) \sim F \qquad \quad\quad\]
\[2.\text{ If  }X \sim F \text{ and } F \text{ is a continuous cdf, then:} \quad U = F(X) \sim \mathcal{U}(0, 1)\]

\vspace{20mm}
Remark: $ U =  F(X)$ is often called the grade of $X$.
\end{frame}


%\begin{frame}
%\frametitle{Motivation}

%It could be interesting (and fun) to find a way to decompose the multivariate cdf $F_{\textbf{X}}$ into the univariate margins $F_{X_1}, ..., F_{X_n}$ and some function describing the dependence among the n variables, preferably stripped of the univariate information.

%\end{frame}


\begin{frame}
\frametitle{Motivation}

Try to decompose the multivariate distribution into:
\begin{itemize}
\item{n univariate margins $F_{X_1}, ..., F_{X_n}$}
\item{a Copula $C$} 
\end{itemize}

\end{frame}

\section{Definition}
\frame{\sectionpage}
\begin{frame}
\frametitle{What is a Copula?}
\begin{figure}[htbp]
%\centering \includegraphics[height=5cm, width=9cm]{Copulasim.pdf}
%\caption{Copula construction illustrated}
\label{copsim}
\end{figure}
\end{frame}


\begin{frame}
\frametitle{What is a Copula?}
\begin{figure}[htbp]
%\centering \includegraphics[height=5cm, width=9cm]{Copulasim2.pdf}
%\caption{Copula construction illustrated}
\label{copsim2}
\end{figure}
\end{frame}

\begin{frame}
\frametitle{What is a Copula?}
\begin{figure}[htbp]
%\centering \includegraphics[height=5cm, width=9cm]{Copulasim3.pdf}
%\caption{Copula construction illustrated}
\label{copsim3}
\end{figure}
\end{frame}

\begin{frame}
\frametitle{What is a Copula?}
A copula is the joint distribution of the grades of our variables of interest. We ``erase'' the information on the marginals by constructing the grades and only information on the structure of the dependence remains.

 \begin{exampleblock}{Definition}
A copula is a function 
\[C: [0, 1] ^n \rightarrow [ 0, 1]: \]
\[(u_1, ..., u_n) \mapsto C(u_1, ..., u_n)\]
such that:
\begin{itemize}
\item{$\forall u \in [0, 1]^n : \quad\exists u_j = 0 :\quad C(u) = 0,\quad j= 1, ..., n$}
\item{$\forall u_j: \qquad C(1, ..., u_j, ..., 1) = u_j \quad(u_1, ..., u_{j-1}, u_{j+1}, ... u_n = 1)$}
\item{C is n-increasing}
\end{itemize}
\end{exampleblock}

\end{frame}

\begin{frame}
\frametitle{Bivariate Case}
In the bivariate case, the properties are easier to parse:

 \begin{exampleblock}{Definition}
A bivariate copula is a function 
\[C: [0, 1] ^2 \rightarrow [ 0, 1]: \]
\[(u_1, u_2) \mapsto C(u_1, u_2)\]
such that:
\begin{itemize}
\item{$\forall u \in [0, 1]^2 :\quad C(u_1, 0) = C(0, u_2) = 0$}
\item{$C(u_1, 1) = u_1$ and $C(1, u_2) = u_2$}
\item{C is 2-increasing, meaning: $C(u_1, u_2)  - C(u_1, v_2) - C(v_1, u_2) + C(v_1, v_2) \geq 0$, for $v_1 \leq u_1$ and $v_2 \leq u_2$}
\end{itemize}
\end{exampleblock}

\end{frame}

\section{Sklar's Theorem}
\frame{\sectionpage}
\begin{frame}
\frametitle{Sklar's Theorem}

What exactly does the relationship between joint cdf, marginals and copula look like?



\end{frame}

\begin{frame}
\frametitle{Sklar's Theorem (bivariate)}

Let $F_{\textbf{X}}$ be a bivariate joint cdf for the RV $\textbf{X} = (X_1, X_2)$ with marginal cdfs $F_{X_1}$ and $F_{X_2}$. Then there exists a copula C such that:

\[F_{\textbf{X}}(x_1, x_2) = C(F_{X_1}(x_1), F_{X_2}(x_2)) = C(u_1, u_2),\]

\vspace{3mm}with $x_1, x_2 \in \bar{\mathbb{R}}$ and $\bar{\mathbb{R}} := \mathbb{R} \cup \{-\infty, \infty\}$


\end{frame}

\begin{frame}
\frametitle{Relating the joint density}
Assuming $F_{\textbf{X}}$ and $C$ are differentiable, calculating the joint density in combination with Sklar's Theorem gives:
\[\frac{\partial^2 F_{\textbf{X}}(x_1, x_2)}{\partial x_1 \partial x_2} = f_{\textbf{X}}(x_1, x_2) = c(F_{X_1}(x_1), F_{X_2}(x_2))f_{X_1}(x_1)f_{X_2}(x_2)\] with \[c(F_{X_1}(x_1), F_{X_2}(x_2)) = \frac{\partial^2 C(F_{X_1}(x_1), F_{X_2}(x_2))}{\partial F_{X_1}(x_1) \partial F_{X_2}(x_2)},\] the copula pdf.


\end{frame}

\begin{frame}
\frametitle{Interpretation}

The copula density can therefore be interpreted as a multiplicative adjustment of the joint density under independence.
The concept easily transfers to higher dimensions in an analogous manner.

\end{frame}

\section{Archimedean Copulas}

\frame{\sectionpage}

\begin{frame}
\frametitle{Classification of Copulas}

Different kinds of copulas can be distinguished, e.g.
\begin{itemize}
\item{Elliptical Copulas, such as the Gaussian}
\item{Archimedean Copulas}
\item{...}
\end{itemize}
\end{frame}


\begin{frame}
\frametitle{Archimedean Copulas}

A copula C is called an Archimedean copula, if it can be written in terms of a continuous generator function $\phi: [0, \infty] \rightarrow [0, 1]$ : 
\[C(u) = \phi (\phi ^{-1}(u_1)+ ... + \phi ^{-1}(u_n)),\]
with $u \in [0, 1]^n$. \\

The generator function $\phi$ holds the properties:

\begin{itemize}
\item{$\phi (0) = 1$}
\item{$\phi (\infty) = 0$}
\item{$(-1)^k \phi ^{(k)}(t) \geq 0,$ for $k \in \{0, 1, ..., n-2\}, t\in (0, \infty)$}
\end{itemize}
\end{frame}


\begin{frame}
\frametitle{Exemplary families of Archimedean Copulas}

Families of Archimedean copulas include:
\begin{itemize}
\item{Clayton, with $\phi(t) = (1 + t)^{-\frac{1}{\theta}},$ with $\theta \in (0, \infty)$}
\item{Frank, with $\phi(t) = -\log{(1-(1- \exp{(-\theta)})\exp{(-t)})}/\theta,$ with $\theta \in (0, \infty)$}
\item{Gumbel-Hougaard, with $\phi(t) = \exp{(-t^{\frac{1}{\theta}})},$ with $\theta \in [1, \infty)$}
\item{Ali-Mikhail-Haq, with $\phi(t) = (1-\theta)/\exp{(t)}-\theta,$ with $\theta \in [0, 1)$}
\item{...}
\end{itemize}

\end{frame}

%------------------------------------------------
\section{Appendix} 
%------------------------------------------------
\frame{\sectionpage}
\begin{frame}
\frametitle{A.1 Proof of transformation property}
Let $X = F_{X}^{-1}(U)$ with $U \sim \mathcal{U} (0, 1)$
\begin{align}
P(X \leq x) &= P(F_{X}^{-1}(U) \leq x) \\ &= P(U \leq F_{X}(x)) \\ &= F_{X}(x) 
\end{align}
Which is the cdf of a uniform distribution.

\end{frame}


\begin{frame}
\frametitle{A.2 Proof of grade property}
Let $X \sim F$ and $F$ be a continuous cdf, then:
\begin{align}
F_{U}(u) &= P(U \leq u) \\ &= P(F_{X}(X) \leq u) \\ &= P(X \leq F_{X}^{-1}(u)) \\ &= F_{X}(F_{X}^{-1}(u)) = u 
\end{align}
Which is the cdf of a uniform distribution.

\end{frame}


\begin{frame}
\frametitle{A.3 n-increasing function}
A function $C: [0, 1]^n \rightarrow [0, 1]$ is n-increasing if for any hyper-rectangle \[R = \prod_{i=1}^{n} [x_i, y_i] \subseteq [0, 1]^n\]
the C-measure / C-volume is non-negative:
\[V_{C}(R) \geq 0\]

\end{frame}


\begin{frame}
\frametitle{A.4 Fréchet-Hoeffding bounds}
For a bivariate copula, the following bounds are called the Fréchet-Hoeffding bounds:
\begin{align}
max(u_1 + u_2 - 1, 0)  &\leq C(u_1, u_2) \leq min(u_1, u_2)
\end{align}
The upper bound is always (not just the bivariate case) a copula itself. The lower boundary only in this case of two variables.

\end{frame}

\begin{frame}
\frametitle{A.5 Origin of the term Archimedean}
The Archimedean axiom for the positive real numbers:
If $a, b$ are positive real numbers, then there exists an integer $n$, such that $na > b$. \\
Since an Archimedean copula behaves like a binary operation on the unit interval (assigns each pair $(u_1, u_2) \in [0, 1]^2$ a value $C(u)$), one can prove that a version of the axiom holds for the Abelian semi-group $(\textbf{I}, C)$.
The enjoyability of the proof is rather limited, therefore we will ignore the details here.

\end{frame}


\begin{frame}
\frametitle{Properties of Archimedean Copulas}

The class of Archimedean copulas is an associative class of copulas, i.e.:
\[C(C(u_1, u_2), u_3) = C(u_1, C(u_2, u_3))\]

\end{frame}

\begin{frame}
\frametitle{Properties of Archimedean Copulas}

The class of Archimedean copulas is a permutation-symmetric class of copulas, i.e.:
\[C(u_1, u_2, u_3) = C(u_3, u_2, u_1)\]

\end{frame}

\begin{frame}
\frametitle{Properties of Archimedean Copulas}

Generators can be easily extended, since $\phi$ being a generator is equivalent to $a\phi$ being a generator, for $a > 0$.

\end{frame}



%------------------------------------------------



%------------------------------------------------




\end{document}